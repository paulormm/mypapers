\documentclass{IEEEconf}
\usepackage[T1]{fontenc}
\usepackage[latin1]{inputenc}
\usepackage{verbatim}
\usepackage{scalefnt}
\usepackage{xcolor}
\usepackage{ulem}
\usepackage{type1cm}
\usepackage{url}
\usepackage{courier}
\newcommand{\TODO}[1]{{\color{red}\textbf{\uwave{#1}}}}

\title{A Study of the Relationships between Source Code Metrics and Attractiveness in Free Software Projects} 

\author{
  Paulo Meirelles, Carlos Santos Jr.,
  \\
  Jo\~ao Miranda, Fabio Kon
  \\
  \begin{affiliation}
    Free Software Competence Center\\
    Institute of Mathematics and Statistics\\
    University of S\~ao Paulo, Brazil\\
    (CCSL-IME/USP)\\
    \email{\{paulormm,denner,joao,kon\}@ime.usp.br}
  \end{affiliation}
  \and
  Antonio Terceiro, Christina Chavez
  \\
  \begin{affiliation}
    Department of Computer Science\\
    Federal University of Bahia, Brazil\\
    (DCC-UFBA)\\
    \email{\{terceiro,flach\}@dcc.ufba.br}
  \end{affiliation}
}

\begin{document}
\normalem
\def\UrlFont{\tt\footnotesize}
\maketitle

\begin{abstract}
A significant number of Free Software projects has
been widely used and considered successful. However, there is an even larger
number of them that cannot overcome the initial step towards building
an active community of users and developers.
%
In this study, we investigated whether there are relationships between source 
code metrics and attractiveness, i.e., the ability of a project to attract
users and developers. To verify these relationships, we analyzed 6,773
Free Software projects from the SourceForge.net repository.
%
The results indicated that attractiveness is indeed correlated to some source
code metrics. This suggests that measurable attributes of the project
source code somehow affect the decision to contribute and adopt a free software project.
%
The findings described in this study show that it is relevant for project leaders
to monitor source code quality, most specifically a few objective metrics, since
these can have a positive influence in their chances of forming a community of
contributors and users around the software, enabling further enhancement in its quality.
\end{abstract}

\section{Introduction}
\label{introduction}

The adoption of Free and Open Source Software\footnote{In our work,
we consider the terms Free Software and Open Source Software (OSS) equivalent.}
has significantly increased in the last decades, to the point of becoming
influential to the global economy~\cite{Benkler06}.
%
Although Free Software has emerged as a movement supported by volunteer developers,
many large companies are now involved in it~\cite{Wasserman2007, Riehle2007}.
%
According to a Forrester Consulting survey, which compared large companies
in Europe and North America~\cite{Forrester-Consulting2008},
the usage of Free Software is currently widespread in the back-end, middleware,
office productivity tools, and business applications software categories.
%
Moreover, this survey states that 92\% of the senior business and IT executives
say that Free Software products have met and, in some cases, exceeded
their quality expectations.

This satisfaction and quality is usually achieved thanks to the collaboration
of a large user and developer community who reports failures,
fixes bugs, and adds features.
%
In fact, the Free Software development model is said to offer two main advantages:
the potential for peer-review and the possibility of attracting developers
from different parts of the world~\cite{Michlmayr2005}.
%
Hence, an important issue for Free Software projects is to attract volunteers~\cite{Stewart2006}.

However, not all Free Software projects reach success and high quality~\cite{Michlmayr2005}.
The amount of inactive projects is undoubtedly higher compared to the number of active projects.
%
To illustrate this scenario, consider the data extracted in November, 2009,
from Sourceforge.net, one of the most popular Free Software repositories.
Out of its 201,494 projects, only 60,642 had more than one
release, 40,228 had been downloaded more than once, and 23,754 had more than one member.
%
Finally, only 12,141 projects matched all these criteria simultaneously.
This may indicate that no more than 6\% of the projects on
SourceForge.net are able to have a healthy community of users
and developers and benefiting from a Bazaar style of development
\cite{CatedralBazzar}.

Santos Jr. \emph{et al.} defined a theoretical model for attractiveness as
a crucial construct for Free Software projects, proposing their
(i) typical origins (e.g., license type, intended audience, type of project, and development status);
(ii) indicators (e.g., number of members and downloads);
(iii) consequences (e.g, levels of activity, efficiency, likelihood of task completion,
time for task completion, and software quality)~\cite{Santos2010}.
%
They suggested that the success of any project depends on
its level of attractiveness to potential contributors and users.
%
Based on this model, our study explored some of the factors that may enable projects to build
a community by attracting users and developers. Specifically, our focus rests on objective factors:
we investigated whether attractiveness can also be influenced by measurable
source code attributes.

Although source code metrics have been proposed since the 1970s~\cite{SEI88},
their potential use as guidelines for software development has not
been fully explored yet~\cite{Tempero}.
%
In particular, we have observed that many Free Software projects do not practice
source code quality evaluation and have no tools available to do so.
%
This lack of systematic code evaluation leaves a lot of room for improvement
in Free Software projects' processes and practices~\cite{Michlmayr2005}.

In general, despite the high importance of source code in the Free Software
community, called the ``show me the code'' culture, source code metrics are
often not perceived as an indicator of quality.
%
To address this apparent contradiction, we argue, theoretically, that source
code metrics are related to a project attractiveness and, thus, influence its success.
%
To verify these ideas empirically, we analyzed 6,773 projects
written in the C language from SourceForge.net.
%
We show that, considering such a sample, one structural complexity metric
and two size metrics play an important role in explaining attractiveness,
here represented by the number of downloads and the number of project members.

%\TODO{must be "we argue" and "we show" in past?}

The remainder of this paper is organized as follows:
%
Section \ref{background} presents the theoretical foundations of source code
metrics and attractiveness.
%
Section \ref{researchDesign} shows the selection criteria and definition for
the variables used in our study.
%
Section \ref{hypotheses} presents our hypotheses and discusses their tests
and results.
%
Section \ref{relatedWork} reviews related work.
%
Conclusions and future research directions are discussed in Section \ref{conclusion}.

\section{Theoretical Background}
\label{background}

We used seven metrics, selected according to criteria shown in 
Section \ref{researchDesign}, to build a statistical model that represents the 
relationships proposed.
%
The concept of attractiveness and its proxies were based on Santos Jr.'s
attractiveness model \cite{Santos2010}.

Firstly, some aspects of software quality can be evaluated by using
objective metrics, and some of these metrics can be collected automatically.
%
For our selected metric set, we used the concept of module as a general term for
referring to the different types of building-blocks used in software
development, depending on the programming language and/or paradigm. We
use the general term ``module'' to stand for classes, abstract data
types, source files, etc. Similarly, we generalized the concept of ``method''
to ``subroutine'', which is a portion of source code that performs a specific task,
in general.

%Source code metrics

The most commonly used metric to measure software size is \emph{Lines of Code} (LOC).
%
Using the LOC metric as a basis for comparison between projects requires
the projects to be written in the same programming language \cite{Jones91}.
LOC indicates the number of non-blank, non-comment source code lines in
the project.

Another useful metric for software size is the \emph{number of modules},
which is somewhat less influenced by programming languages and line-level
coding styles. Therefore, it may be used to compare projects written in
different languages \cite{Tempero}.

When considering characteristics such as maintainability, flexibility,
comprehension effort, and source code quality in general, one has to take into
account not only the size metrics described above but also structural metrics,
such as the ones described below.

\emph{Number of Methods} is used to measure class size in terms of
the operations it supports. A class should not have an excessive number of operations \TODO{citacao?}.
%
This metric is used to help identify the reuse potential of a class.
In general, classes with a large number of operations are more difficult
to reuse because they tend to be less cohesive \cite{Lorenz94}.

\emph{Number of Public Attributes} (NPA) and \emph{Number of Public Methods} (NPM)
are metrics related to class encapsulation. They measure the potential
communication among classes \cite{Bansiya97}.
% 
Object-orientation best practices recommend that class attributes be only
manipulated via accessor methods \cite{beck97}. Thus, class attributes should be private,
indicating that the optimal number for this metric is zero. 

\TODO{esta OK falar sobre orientacao a objetos, dado que so analisamos projetos em C?}
%
The number of public methods in a class represents the size of class interface.
Methods are directly related to the functionality provided by the class.
High values for this metric indicate that a class has a lot of functions,
which also contradicts the recommended best practices in object-orientation \TODO{citacao?}.

\emph{Cohesion} is a measure of the diversity of ``topics'' that a
module implements. High cohesion values indicate modules that focus on a
single aspect of the system, while low cohesion indicates modules that
deal with several different aspects. Highly cohesive modules are easier to
understand, maintain, and modify.
%
A metric commonly used for cohesion is \emph{Lack of Cohesion on
Methods} (LCOM), originally proposed by Chidamber and Kemerer \cite{Chidamber94}.
High LCOM values indicate low cohesion, while low LCOM values indicate high cohesion.

The first LCOM definition by Chidamber and Kemerer, called LCOM1, has
received a lot of criticism and revision proposals. LCOM1 corresponds to
the number of pairs of methods of a class using the same attributes.
%
In this study, we used the revised definition by Hitz and Montazeri,
known as LCOM4 \cite{LCOM4}.
%
LCOM4 indicates the number of components of an undirected graph, where the
nodes are the methods. When two methods use at least an attribute in \TODO{use = have?}
common, there is an edge between the corresponding nodes. Additionally, the graph
has an edge between these methods, if one invokes the other.


\emph{Coupling} is a measure of how one module is connected to other modules
in the project. 
%
High coupling indicates a greater difficulty to change the
modules of the system, since a change in one module may have an impact in
all other modules that are coupled to it.
%
In other words, if coupling is high, the software tends to be less flexible,
more difficult to adapt, and more difficult to understand.
%
An objective metric for measuring coupling is \emph{Coupling Between Objects} (CBO), 
also proposed by Chidamber and Kemerer. CBO measures how many modules are
used by the module being analyzed \cite{Chidamber94}.

The more complex a piece of software, the more challenging it is to change and
evolve it. Coupling and cohesion have been described and discussed in
many works as essential indicators of structural complexity \cite{darcy2005}.
%
In the field of Object-Oriented Programming, it is widely known that to build 
high-quality and flexible software, it is advisable to seek low coupling and 
high cohesion \cite{richter99}.

In fact, Darcy \emph{et al.} \cite{darcy2005} have shown that, individually, neither
coupling nor cohesion are related to software maintenance effort, 
they are metrics that must be considered together. When combined, the 
product of coupling and cohesion as a metric is positively correlated to 
the maintenance effort.
% 
Therefore, we use the product of coupling (CBO) and cohesion (LCOM4) as
our metric of structural complexity (SC), as in \cite{darcy2005}.

%Attractiveness

Secondly, attractiveness is the capacity of bringing users and developers to a project.
%
A Free Software project is attractive as much as it has the ability to interest 
potential users and developers, who will later use the software and, ultimately, 
participate on tasks to improve the project \cite{Santos2010}.

In our study, we measured this concept based on two empirical indicators.
%
Attractiveness can be estimated by the number of people that
joined the project and the number of software downloads.
%
These indicators correlate to project developers and software users, respectively.
%
Before a Free Software project can receive failure reports, bug fixes, 
and new features, it must be attractive to volunteers, who normally first join the project 
and later provide contributions.
%
Over time, these contributions affect the number of downloads and bring more 
members, creating a positive feedback loop.

One should note that the number of downloads and number of members 
at SourceForge.net are simply proxies to the actual number of users and number of 
developers in the project, respectively.
%
The study explored a large number of projects and applied the same
criteria uniformly to all of them. The number of downloads is a
representation of the number of people interested in the software. The
number of members represents the number of contributors to the project.

\section{Research Design} 
\label{researchDesign}

We analyzed the Free Software projects from SourceForge.net that are written in 
C and that had been downloaded more than once. The reason to adopt this filter 
was to eliminate projects that do not have an actual software under development.
%
Our goal was to characterize them with respect to structural complexity,
lines of code, number of modules, number of downloads, and number of members.
%
The choice of programming language and metrics for this study were based on the 
currently mature features of \texttt{Analizo}, a source code analysis tool our 
group is developing.

\subsection{The Analizo Tool}

\texttt{Analizo}\footnote{\url{softwarelivre.org/mezuro/analizo}}
is a source code analysis tool. Its architecture was designed in
a way to let it parse source code written in several languages and report useful
information about it.

A basic requirement of our source code analysis tool was the ability to analyze 
source code written in multiple languages.
%
Most existing tools use object code to extract data, making it impossible to process 
projects that do not compile due to failures in either the source or 
in its dependencies \cite{hassan05}. In addition, object-based tools are not 
capable of analyzing features present only in the source, such as comments. 
%
To avoid these problems, \texttt{Analizo} is designed to extract the information
directly from source code parsers.

In the current version, to parse the source code \texttt{Analizo} uses 
\texttt{Doxyparse}\footnote{\url{softwarelivre.org/mezuro/doxyparse}}, 
a multi-language source code parser based on Doxygen's internals.
%
In theory, it is able to parse all the languages supported 
by Doxygen\footnote{\url{doxygen.org}},
but, up to the moment, it has been tested only with C, C++, and Java source code.

At that time, \texttt{Analizo} was fully tested with C source code downloaded 
from Sourceforge.net projects, providing the following metrics:
%
Afferent Connections per Class (ACC),
Coupling between Objects (CBO),
Coupling Factor (COF),
Depth of Inheritance Tree (DIT),
Lack of Cohesion (LCOM4),
Lines of Code (LOC),
Lines per Method (AMZ\_Size),
Number of Attributes (NOA),
Number of Children per Class (NOC),
Number of Methods (NOM),
Number of Modules/Classes (NM),
Number of Public Attributes (NPA),
Number of Public Methods (NPM),
and Response for Class (RFC).

The correctness of the metrics computation was evaluated by comparing the 
results provided by Analyzo and other existing tool such as
%
CCCC\footnote{\url{cccc.sourceforge.net}}, 
Cscope\footnote{\url{cscope.sourceforge.net}},
Eclipse-Metrics\footnote{\url{metrics.sourceforge.net}}, 
and Macxim/Spago4Q\footnote{\url{qualipso.dscpi.uninsubria.it/macxim}}.
%
Also, we checked other existing extractors such as LDX, CTAGX, and CPPX \cite{hassan05}.

For this study, we used NM, NPA, NPM, LCOM and CBO \TODO{LCOM4?}. They are metrics originally
proposed for metrics for the object-oriented paradigm in mind.  
%
Nevertheless, if we generalize the concepts of ``class'' and ``method'' 
to ``module'' and ``subroutine'', respectively, we can also apply them to 
other programming language paradigms.

\subsection{Sample and Data Collection}

SourceForge.net shares its data to support Free Software researchers.
%
In this study, we used the data available in a database managed by 
the University of Notre Dame\footnote{\url{nd.edu/~oss/Data/data.html}} 
and another one provided by the FLOSSMole project\footnote{\url{flossmole.org}}. 
%
We accessed these databases in November, 2009 and collected data about 
all the projects that matched the following criteria:
%
\begin{itemize}
\item \emph{Written in the C language}, because, at that time, \texttt{Analizo} 
was limited to analyzing C source code and there are few academic works 
that evaluate a large number of Free Software projects written in C;
%
\item \emph{more than once download}, because projects with no downloads are 
probably either non-development projects, or projects that have just started, 
or are other special cases.
\end{itemize}

This provided us with a list of 11,433 projects. We developed a script
to download the code of all of them by checking the ``files'' section
in the SourceForge.net project pages.
%
After this step, we obtained the source code for 10,128 projects as
some of them had no available files.

Later, we developed another script that runs \texttt{Analizo} sequentially 
for all projects and stores the computed metrics in a single database.
%
We successfully computed the metrics for 6,773 projects, because 
(i) some downloaded files did not contain source code (e.g., binary-only
downloads),
%
(ii) the source code was not written in C, (the project was
incorrectly classified as being written in C), or
%
(iii) they could not be processed by \texttt{Analizo} due to severe
errors in the source code (e.g., infinite loops).

Finally, we cross-joined the two datasets (the SourceForge.net data available
from the University of Notre Dame and FLOSSMole on the one side and the source 
code metrics calculated by \texttt{Analizo} on the other side) so that 
we could perform the needed statistical analysis. The complete data set used 
for this study is available on the Web\footnote{\url{ccsl.ime.usp.br/mangue/data}}.

\subsection{Variables}

Among the fourteen source code metrics that \texttt{Analizo} provided, we
selected seven for our initial analysis: average of CBO, average of LCOM4, 
total of LOC, average of NOM, total of NM, average of NPA, and average of NPM. 

\vspace{-3em}
\begin{center}
\begin{table*}[hbt]
\scalefont{0.9}
\centering \caption{Parametric Correlations: Pearson}
\begin{tabular}{|l|c|c|c|c|c|c|c|c|c|c|} \hline

\textbf{Variable} & CBO   & LCOM4 & SC    & NM    & LOC   & NPA   & NOM   & NPM   & Mbrs & DLs 	
\\ \hline
CBO               & -     & 0.141 & 0.723 & 0.380 & 0.608 & 0.423 & 0.434 & 0.492 & 0.113 & 0.129  	
\\ \hline
LCOM4             & 0.141 & - 	  & 0.786 & 0.019 & 0.472 & 0.102 & 0.311 & 0.361 & 0.080 & 0.107
\\ \hline
SC		  & 0.723 & 0.786 & - 	  & 0.254 & 0.666 & 0.338 & 0.493 & 0.564 & 0.127 & 0.156
\\ \hline
NM           	  & 0.308 & 0.019 & 0.254 & - 	  & 0.799 & 0.730 & 0.815 & 0.827 & 0.311 & 0.344
\\ \hline
LOC               & 0.608 & 0.472 & 0.666 & 0.799 & - & \textbf{0.872} & \textbf{0.923} & \textbf{0.927} & 0.328 & 0.410
\\ \hline
NPA 		  & 0.423 & 0.102 & 0.338 & 0.730 & \textbf{0.872} & - & 0.756 & 0.761 & 0.303 & 0.386
\\ \hline
NOM               & 0.434 & 0.311 & 0.493 & 0.815 & \textbf{0.923} & 0.756 & - & 0.886 & 0.320 & 0.380
\\ \hline
NPM 		  & 0.492 & 0.361 & 0.564 & 0.827 & \textbf{0.927} & 0.761 & 0.886 & - & 0.308 & 0.365
\\ \hline
Mbrs		  & 0.113 & 0.080 & 0.127 & 0.311 & 0.328 & 0.303 & 0.320 & 0.308  & - & 0.676
\\ \hline
DLs		  & 0.129 & 0.107 & 0.156 & 0.344 & 0.410 & 0.386 & 0.380 & 0.365 & 0.676 & -
\\ \hline
\end{tabular}
\label{pearson}
\scalefont{1}
\end{table*}
\end{center}


Firstly, since COF, DIT, NOC, RFC, and ACC (used to calculate COF) are metrics that 
originally offer information about source code attributes in 
Object-Oriented paradigm, there is a need to adaptation them to the procedural paradigm. 
%
Nevertheless, in this first study, we did not try to generalize its concepts for
a non-Object-Oriented paradigm because we did not want to investigate the 
source code attributes that they provide. \TODO{isso aqui tem explicar melhor. parece que vc vai dizer pq nao adotamos mas acaba nao dizendo}

Secondly, \emph{lines per method} (AMZ\_Size) and \emph{lines of code} (LOC) 
measure number of lines. However, AMZ\_Size is a module level metric and total of
LOC is a global level metric that include LOC of methods, and so AMZ\_Size effect is captured through LOC. Thus, 
we did not use AMZ\_Size in our statistical analysis.
%
Additionally, we were only interested in \emph{public attributes} (NPA) and 
we did not use \emph{number of attributes} (NOA), because, in this study, 
we analyzed information on a project level (sum or average of metrics for all modules).
%
In any case, both, AMZ\_Size and NOA, can give important information when a module is 
individually analyzed.



\begin{comment}

\begin{center}
\begin{table*}[hbt]
\scalefont{0.9}
\centering \caption{Non-Parametric Correlations: Spearman}
\begin{tabular}{|l|c|c|c|c|c|c|c|c|c|c|} \hline

\textbf{Variable} & CBO   & LCOM4 & SC    & NM    & LOC   & NPA   & NOM   & NPM   & Mbrs & DLs 
\\ \hline
CBO               & - & 0.340 & 0.803 & 0.473 & 0.662 & 0.523 & 0.518 & 0.566 & 0.156 & 0.169
\\ \hline
LCOM4		  & 0.340 & - & 0.773 & 0.213 & 0.490 & 0.348 & 0.478 & 0.516 & 0.129 & 0.162
\\ \hline
SC		  & 0.803 & 0.773 & - & 0.370 & 0.685 & 0.478 & 0.571 & 0.631 & 0.164 & 0.196
\\ \hline
NM		  & 0.473 & 0.213 & 0.370 & - & 0.793 & 0.718 & 0.818 & 0.828 & 0.284 & 0.320
\\ \hline
LOC		  & 0.662 & 0.490 & 0.685 & 0.793 & - & \textbf{0.863} & \textbf{0.918} & \textbf{0.922} & 0.307 & 0.392
\\ \hline
NPA 		  & 0.523 & 0.348 & 0.478 & 0.718 & \textbf{0.863} & - & 0.758 & 0.765 & 0.280 & 0.363
\\ \hline
NOM		  & 0.518 & 0.478 & 0.571 & 0.818 & \textbf{0.918} & 0.758 & - & 0.895 & 0.300 & 0.362
\\ \hline
NPM		  & 0.566 & 0.516 & 0.631 & 0.828 & \textbf{0.922} & 0.765 & 0.895 & - & 0.288 & 0.347
\\ \hline
Mbrs		  & 0.156 & 0.129 & 0.164 & 0.284 & 0.307 & 0.280 & 0.300 & 0.288 & - & 0.598
\\ \hline
DLs		  & 0.169 & 0.162 & 0.196 & 0.320 & 0.392 & 0.363 & 0.362 & 0.347 & 0.598 & -
\\ \hline
\end{tabular}
\label{spearman}
\scalefont{1}
\end{table*}
\end{center}

\vspace{-2em}

\begin{center}
\begin{table*}[hbt]
\scalefont{0.9}
\centering \caption{Non-Parametric Correlations: Kendall}
\begin{tabular}{|l|c|c|c|c|c|c|c|c|c|c|} \hline

\textbf{Variable} & CBO   & LCOM4 & SC    & NM    & LOC   & NPA   & NOM   & NPM   & Mbrs & DLs 
\\ \hline
CBO               & - & 0.244 & 0.650 & 0.341 & 0.483 & 0.377 & 0.373 & 0.410 & 0.118 & 0.114
\\ \hline
LCOM4             & 0.244 & - & 0.597 & 0.148 & 0.341 & 0.240 & 0.333 & 0.362 & 0.097 & 0.109
\\ \hline
SC       	  & 0.650 & 0.597 & - & 0.262 & 0.497 & 0.345 & 0.413 & 0.460 & 0.124 & 0.132
\\ \hline
NM           	  & 0.341 & 0.148 & 0.262 & - & 0.605 & 0.546 & 0.641 & 0.648 & 0.217 & 0.219
\\ \hline
LOC               & 0.483 & 0.341 & 0.497 & 0.605 & - & \textbf{0.660} & \textbf{0.763} & \textbf{0.771} & 0.233 & 0.270
\\ \hline
NPM 		  & 0.377 & 0.240 & 0.345 & 0.546 & \textbf{0.660} & - & 0.588 & 0.596 & 0.213 & 0.249
\\ \hline
NOM		  & 0.373 & 0.333 & 0.413 & 0.641 & \textbf{0.763} & 0.588 & - & 0.864 & 0.228 & 0.248
\\ \hline
NPM		  & 0.410 & 0.362 & 0.460 & 0.648 & \textbf{0.771} & 0.596 & 0.864 & - & 0.219 & 0.237
\\ \hline
Mbrs		  & 0.118 & 0.097 & 0.124 & 0.217 & 0.233 & 0.213 & 0.228 & 0.219 & - & 0.471
\\ \hline
DLs		  & 0.114 & 0.109 & 0.132 & 0.219 & 0.270 & 0.249 & 0.248 & 0.237 & 0.471 & -
\\ \hline
\end{tabular}
\label{kendall}
\scalefont{1}
\end{table*}
\end{center}
\vspace{-2em}

\end{comment}

Nevertheless, in the first stage of our statistical analysis, LOC, NOM, NPA, 
and NPM metrics showed high correlation between each other, according to Pearson's 
parametric correlations shown in the Tables \ref{pearson}.
%
Therefore, since all these metrics represent a similar concept, we selected 
one of them for our statistical analysis to reduce multicollinearity.

Also we analyzed the Spearman and Kendal non-parametric correlations, given that 
some of our variables are not normally distributed.
% 
In our analysis, we observed that after transforming our variables in their logarithmic form, Pearson correlations perform just as well as the non-parametric indices. Thus, we chose the Pearson parametric correlation because it represents the most commonly used form of correlation index, and it also provides the basis for the regression analysis we perform later. That way, we can maintain consistency.

Accordingly, we ended up considering LOC and NM as size metrics. Theoretically, \TODO{essa frase ta meio perdida aqui e comeca com um accordingly que nao faz muito sentido}
the more LOC, the more NM. 
%
However, it is possible to have more lines of code without having more modules, by 
adding code into existing modules. Also, a software could have more modules 
but keep the number of lines of code when refactoring is applied. 
%
In conclusion, they measure different kinds of size metrics: more or less 
influenced by programming languages and coding styles, respectively.
%
Thereby, we collected both metrics.

Furthermore, we understood that \emph{number of modules} (NM) did not highly 
correlate with the others, since we considered a high correlation when the 
Pearson's correlations' values were approximately 0.900 or higher, 
emphasized in Table \ref{pearson}.
%
This indicates that these metrics represent (partly) the same attributes, which did not 
occur with NM when compared to others, using our criteria.

Finally, CBO and LCOM4 were multiplied to obtain the value of our structural 
complexity metric (SC), explained in Section \ref{background}. 
%
These three metrics did not show high correlations with other metrics.
%
As expected, SC showed a direct correlation with both. However, it was not \TODO{reveja esse direct. o que vc quer dizer, positiva?}
as high as the others were, because CBO and LCOM4 had a low correlation 
with each other. This means that SC, statistically, represents different attributes 
when compared to CBO and LCOM4. Thus, endorsing to the theory that CBO and LCOM4 together
offer other information. \TODO{nao entendi o other. =different?}

In summary, according to what was introduced in Section \ref{background} and explained in this section,
our multiple regression model ended up with the following variables:

\begin{itemize}
\item \textbf{Independent variables (source code metrics)}:
%
\begin{itemize}
\item \emph{Structural Complexity (SC)}: The product of CBO and LCOM4 metrics 
%
\item \emph{Lines of Code (LOC)}, the sum of lines of code in all modules of the
project;
%
\item \emph{Number of Modules (NM)}, the total number of all modules of the project.
\end{itemize}

\item \textbf{Dependent variables (attractiveness)}:
\begin{itemize}
\item \emph{Number of Downloads}: a proxy for the number of users of the
project;
%
\item \emph{Number of Members}: a proxy for the number of developers in the
project.
\end{itemize}
\end{itemize}

The model developed in this study revolves around attractiveness,
aiming at the explanation of its causes.
%
We defined a multiple regression model that has attractiveness as its dependent 
variable. It was measured through two indicators: number of downloads and number of members. Thus, we have two different regressions, one for each attractiveness indicator.
%
They are the variables explained by the source code attributes proposed in our hypotheses. 
%
Consequently, SC, LOC and MN metrics, which represent the source code attributes, 
are the independent variables; the influencers of attractiveness.

\section{Research hypotheses}
\label{hypotheses}

Attractiveness is a human perception \cite{Santos2010}. Many factors
may influence a user to download a piece of software, such as a
recommendation from a friend, a hint from a magazine, a fancy Web site, work
needs, or new features.
%
Similarly, the decision of a developer to support a Free Software project involves
several subjective matters such as favorite programming language/paradigm and 
potential user community size, for example.

In this study, we investigated whether some attributes obtained via 
source code quality metrics might influence the attractiveness of a Free 
Software project, and whether both may interact with each other to influence 
one's perception of quality.
%
Accordingly, we formulated three hypotheses:

\emph{H1: Free Software projects with higher structural complexity have
lower attractiveness.}
%
The higher the software complexity, the more difficult
it is to understand its source code for maintenance and evolution purposes.
This leads to an increase in the maintenance effort, and makes it more
difficult to attract new members and users for the project. Over time, with
less members and users, the project may lose its ability to add new features
and fix bugs and, consequently, its ability to evolve and meet the user's
changing requirements.

\emph{H2: Free Software projects with more lines of code have
higher attractiveness.}
%
To some extent, lines of code reflect the amount of features of the project and
the amount of work that have been put into it. Therefore, projects with
more lines of code will usually attract more users (since they have more
features) and developers (since they offer more opportunities for
contribution).

\emph{H3: Free Software projects with a higher number of modules have
higher attractiveness.}
%
The number of modules may indicate the project size and the possibility
of working in parallel in independent modules. More modules may indicate a
concern with good design, with better modularization, which facilitates contributions. 
This attracts more members, who can
write more features and fix more bugs, which would then attract more users.

%\TODO {Desenvolver e explicar melhor as duas ultimas hipotese. 
%       Colocar uma poss�vel/esperada conclusao sobre ela.
%       Discutir com Terceiro.}

In summary, our hypotheses were based on the idea that most developers of a Free Software project 
are interested in working with quality code, so they can learn 
from it and more easily contribute to its development.
%
Conversely, users are typically interested in features that suit their needs. 
%
Many of the first successful Free Software projects, such as Emacs, GCC, and 
later, Linux, benefited from being well-written pieces of software whose main 
users were software developers themselves.


\subsection{Hypotheses testing}

We specified a multiple regression model to explain the relationships
between the
selected source code metrics and attractiveness in Free Software projects. 
%
Before running this model, we analyzed the descriptive statistical values 
of our dataset, presented in Table \ref{table:statistics}.
%
It shows the sample size (N), natural values of minimum, maximum, mean, 
and standard deviation for each variable, indicating the characteristics of our sample.


First, we analyzed the variables in their natural form to verify their
distribution.
%
The Skewness and Kurtosis probability distribution showed high values, indicating non-normality. Thus,
we transformed the variables to a logarithm scale for linearization, 
which reduced the Skewness and Kurtosis values and made them proper to run multiple regressions \cite{Crowston2002}.
%
The arithmetic mean and standard deviation of the logarithm values can also be seen
in Table \ref{table:statistics}.

\begin{center}
\begin{table*}[hbt]
\scalefont{0.9}
\centering \caption{Descriptive statistics}
\begin{tabular}{|l|r|r|r|r|r|r|} \hline
  & \multicolumn{4}{|c|}{Raw} & \multicolumn{2}{|c|}{Logarithm}\\ \hline

\textbf{Metric} & Min. & Max.     & Mean       & Std. Deviation  & Mean & Std. Deviation \\ \hline

CBO             & 0       & 7.12        & 2.26       & 9.04            & 0.35 & 0.98 \\ \hline

LCOM4           & 0       & 2.62        & 4.77       & 1.20            & 1.01 & 1.09 \\ \hline

SC      	& 0       & 4,940       & 15.79      & 114.69          & 1.37 & 1.57  \\ \hline

LOC        	& 11      & 2,983,103   & 17,700.00  & 91,614.70       & 8.28 & 1.58 \\ \hline

NM     		& 1       & 7,177       & 74.98      & 276.54          & 3.08 & 1.39 \\ \hline

Mbrs           	& 1       & 288         & 2.90       & 6.19            & 0.59 & 0.79  \\ \hline

Dls          	& 6       & 9,000,000 & 957,000.00 & 17,760,000.00   & 8.20 & 2.66  \\ \hline

\end{tabular}
\label{table:statistics}
\scalefont{1}
\end{table*}
\end{center}
\vspace{-3em}


\begin{center}
\begin{table*}[hbt]
\scalefont{0.9}
\centering \caption{Equations and Pearson Correlations}
\begin{tabular}{|l|r|r|r|r|r|r|r|r|} \hline

& \multicolumn{4}{|c|}{Downloads} & \multicolumn{4}{|c|}{Members}\\ \hline
  
\textbf{Metric} & $\beta$ & Std. $\beta$ & T-value & P-value  & $\beta$ & Std. $\beta$ & T-value & P-value
\\ \hline
(Constant)      &  1.551  &  - 	    & 6.12    & \textless 0.001 & -0.668 &  -          & -8.47   & \textless 0.001
\\ \hline
SC (log) 	& -0.286  &  -0,150 & -8.616  & \textless 0.001 & -0.033 &  -0.058     & -3.238  & 0.001
\\ \hline
LOC (log)       &  0.856  &   0.506 & 18.624  & \textless 0.001 &  0.126 &   0.249     & 8.846  & \textless 0.001 
\\ \hline
NM (log)        &  0.008  &   0.004 & 0.186   & 0.852           &  0.087 &   0.148     & 6.625  & \textless 0.001
\\ \hline
   \hline
$R$                               & \multicolumn{4}{|c|}{0.425} & \multicolumn{4}{|c|}{0.348}\\ \hline
$R^2$                             & \multicolumn{4}{|c|}{0.180} & \multicolumn{4}{|c|}{0.121}\\ \hline
\end{tabular}
\label{table:regression}
\scalefont{1}
\end{table*}
\end{center}
\vspace{-2em}

Later, we tested our hypotheses according to our statistical multiple regression model.
%
Table \ref{table:regression} summarizes the regression results based on the Pearson's correlation values.
%
These statistical results indicated a linear dependency between our
source code metrics and each attractiveness variable.

In Table \ref{table:regression}, $\beta$ is a coefficient that indicates the
size of the influence of each metric on each attractiveness indicator.
%
As we can see in Table \ref{table:regression}, lines of code is more strongly 
correlated to downloads and members than structural complexity and number of modules, 
according to the standardized beta (\emph{Std. $\beta$}). Standardized betas are calculated to perform comparisons between variables that are measured using different scales (e.g., lines of code and structural complexity). One cannot compare regular beta coefficients without first standardizing them.
%
Moreover, structural complexity has a negative correlation with attractiveness, 
as expected. Noteworthy is that T-test and P (probability) values represent whether a source code 
metric is a statistically significant predictor or influencer of attractiveness indicators.
%
For downloads, the number of modules is not significant because 
its P-value is greater than 0.05. Finally, in the last line of
Table \ref{table:regression}, R-squared values indicate the percentage of attractiveness (users and developers) variance that
this set of source code metrics is capable of explaining. So, roughly speaking, a R-squared of 20 percent indicates that a set of predictors can explain 20 percent of a dependent variable.
%
We obtained the following equations:

%\vspace{-1.5em}
\begin{displaymath}
\scalefont{0.9}
\begin{array}{lll}
downloads = & 1.551 - 0.286\times log(SC) \\
            & + 0.856\times log(LOC) + 0.008\times log(NM) \\
\\
members   = & -0.668 - 0.033\times log(SC) \\
            & + 0.126\times log(LOC) + 0.087\times log(NM)
\end{array}
\scalefont{1}
\end{displaymath}
%\vspace{-1em}

Each equation has one $R$-value. The coefficient ($\beta$) of each variable is
the size of influence that the independent variable (one of the source code metrics)
has on the dependent variable (attractiveness). So, one unit change in an independent variable generates a beta-size influence on the dependent variable, on average.
%
The $R$-value represents the amount of the dependent variables that can be 
explained by that set of independent variables.
%
In our analysis, the $R$-value indicated that source code metrics explain 18\% ($R^2 = 0.180$) of the number of downloads and 12\% ($R^2 = 0.121$) of the number of members. These are significant values for the social context that an adoption or volunteering of a free software projects are involved.

The data analysis supports our first hypothesis (\emph{ H1: Free Software
projects with higher structural complexity have lower attractiveness.}).
In fact, structural complexity has a negative influence on
attractiveness.
%
When related to downloads, it presents a -0.286 $\beta$ coefficient
and $p < 0.001$.
%
This means that structural complexity has an statistically significant impact on user interest.
In the Free Software context, structural complexity may indicate the
difficulty to make improvements to the software, such as new features and bug
fixes.
%
So, most users may loose interest in the software because another
project may have a greater capacity to meet their evolving needs.
%
A smaller number of users, generating less reports could lead to less bug fixes and
new features, which in turn could lead to less users in the future.
%
When related to members, structural complexity presents a $\beta$ of
-0.033 with $p = 0.001$, indicating that developers avoid to join
projects with high structural complexity. A more complex source code is
more difficult to understand and, consequently, to change. This may prevent
new developers from joining the project. With fewer members, the community around
a project is less active.

%\TODO{usar with ou and quando citar beta e p?}

% H2
The second hypothesis (\emph{H2: Free Software projects with more
lines of code have higher attractiveness}) is also supported by our data.
Lines of code has a positive influence on attractiveness.
%
For downloads, this metric has $\beta = 0.856$, with $p<0.001$.
In this context, lines of code can be an indication of the amount of
software features and amount of work that have been put into the project so far.  
The more features available, the more users
will become interested in the project.  This may make the software more
famous and more useful, attracting new members and users.
%
In addition, lines of code in relation to number of members indicated that
developers are interested in larger projects. The $\beta$ coefficient of
this metric for members is 0.126 ($p<0.001)$. For both downloads and
members, lines of code is the metric with the highest influence, because
it is associated with software features and project size.

% H3
The most interesting results were related to our third hypothesis
(\emph{H3: Free Software projects with a higher number of modules have
higher attractiveness}).
% part 1 - discuss the statistics
For downloads, the data does not support the hypothesis: the high
$p$-value ( $p= 0.852$) does not allow us to claim that the number of
modules has any influence on the number of downloads.  For members, on
the other hand, the hypothesis is confirmed: the number of modules
influences the number of members with $\beta = 0.087$, and $p<0.001$, which is
statistically significant.

% part 2 - discuss why the results are different for users and developers
Both lines of code and number of modules are metrics that represent software 
size. The fact that both influence number of members, but only lines of codes
influence the number of downloads makes us wonder whether they represent
different characteristics of software size.
%
In this context, lines of code probably is related to the amount of features in the project, which helps to attract both users and developers to the project. When lines of
code is kept constant, different values in number of modules represent
different ways of organizing these features over different modules. A
higher number of modules thus indicates a higher modularity, which in turn makes
it easier for developers to work on the project, requiring less
coordination effort. It is easier to work on
modular software.
%
For users, on the other hand, it is probably the case that it does not matter 
whether the software is modular or not; they are only interested in the provided features.

Collaborators in Free Software projects often start participating in the project as
users, attracted by the software features. After that, those users
who have the potential to become developers may begin to contribute code.
%
While a high number of lines of code (and thus of features) is
enough to attract users, project leaders should pay attention to source
code quality. To turn users into developers, the project has to provide
a source code that is easy to understand and modify by keeping structural
complexity as low as possible and modularity at a good level.

\section{Related Work}
\label{relatedWork}

Large Free Software projects such as Debian GNU/Linux, GNOME, and KDE have invested in the 
creation of dedicated teams for quality assurance. These efforts involve everything
from removing bugs and obsolete components to the definition of standards
and strategies to prevent bugs and improve quality \cite{Michlmayr2005}.
However, most projects do not have the resources to have a dedicated quality
team. 
%
Michlmayr \emph{et al.} \cite{Michlmayr2005} performed a study on quality
assurance problems in Free Software such as unsupported code, configuration management,
security updates, users not knowing how to report bugs, the difficulty in
attracting volunteers, lack of documentation, and problems with coordination
and communication. None of these problems, however, are related to the quality
of the source code \emph{per se}.

Barkmann \emph{et al.} \cite{Barkmann2009} analyzed 146 Free Software projects
written in Java, identifying the correlation between a set of object-oriented
metrics and their theoretical typical values. \TODO{typical=ideal?}
%
However, in their work the values of source code metrics were not associated 
with problems or attractiveness of Free Software projects.

Stamelos \emph{et al.} \cite{Stamelos2002} presented empirical results on the relationship
between the size of application components and the delivered quality
measured as user satisfaction. 
%
Quality characteristics of 100 applications written for GNU/Linux were compared to 
industrial standards. The results indicated that the so-called structural 
quality (e.g., component size) of an application is related to user satisfaction.

Midha \cite{midha2008} analyzed 450 projects from SourceForge.net and
verified that high values of MacCabe's Cyclomatic Complexity and
Haltead's Effort (complexity metrics) are positively correlated
with the number of bugs and with the time needed to fix bugs. 
%
These metrics were also found to be negatively correlated with
contributions from new developers, i.e., more complex code is less
likely to attract new developers. However, Midha's study used complexity metrics
measured at the subroutine level, while in our study we use
complexity metrics at the module level.

Capra \emph{et al.} \cite{capra2008} have shown that open governance is associated 
with higher software design quality on a study with 75 Free Software projects. 
They defined software design quality in terms of 5 Object-Oriented metrics,
of which only CBO is used in our study. 
%
An open governance structure together with the lack of formal management and 
strict deadlines enables developers to enhance software design to have a 
high-quality product, since they do not suffer pressure to release the software. \TODO{achei meio perdida essa frase. pelo menos adicionar uma citacao seria bom}
%
Moreover, a better software design fosters a more open governance by allowing
developers to work in independent modules without the need for
explicit coordination activities. However, Capra's study has not addressed the
issue of attractiveness.

Bargallo \emph{et al.} \cite{bargallo2008} analyzed 56 Free Software projects, 
studying the relationship between software design quality and project success. 
%
They defined success in terms of downloads, page views and development activity,
and design quality in terms of the object-oriented metrics CBO, DIT, MIF, and NOC.
%
They found that the most successful projects exhibited lower design
quality. They argue that perhaps in successful projects the main
developers tend to shift their attention to lateral activities, such as
replying to users in forums, instead of focusing on enhancing the code quality.
%
Our results seemed to contradict theirs, but this is not the case. First, their
conceptualization of success is different from our conceptualization of 
attractiveness. 
%
Moreover, we considered structural complexity in terms of CBO 
and LCOM4 metrics together, while they used a different set of metrics to 
represent the notion of design quality, having only CBO in common with 
the present study. Therefore, a straightforward comparison between their study and ours is not so simple.

\section{Conclusion}
\label{conclusion}

A systematic review of 63 empirical studies showed that there is little research
addressing the characteristics or properties of Free Software projects, 
such as their quality, growth, and evolution \cite{Stol2009}.
%
Our study contributes with an unprecedented analysis of
source code metrics from thousands of Free Software projects,
causally linking software source code characteristics with attractiveness.
%
In doing so, we expect to raise awareness on an important topic so far
neglected. Free Software projects fail when they lack attractiveness.
Therefore, understanding what influences attractiveness provides
managerial knowledge to project leaders, pointing them to the right
direction on prioritizing their resources.

Our results indicated that source code size and structural complexity
explain a relevant percentage of the attractiveness of Free Software projects.
%
Attractiveness is based on human perceptions and influenced by people's
cognition, making it a complex issue, hard to understand and explain
completely. 
%
Nevertheless, our study was able to explain  18\% of software 
users and 12\% of project developers, through a set of four source code metrics. 
%
These statistical results are significant for the social context that 
Free Software projects adoption and volunteering are inserted.

In this study, we showed that lines of code (LOC)
has a significant effect on the number of project users.
%
Our results also indicated that structural complexity (SC) has a negative 
influence on project attractiveness.
%
Therefore, a project will face greater difficulties to grow without observing
some source code attributes such as cohesion, coupling, and modularity,
which favors developer contributions such as new features and bug fixes.

In other words, our analysis indicated that software structural complexity growth 
may decrease the positive effects of new added features on attractiveness. 
%
Ideally, a project should keep its complexity constant as new code is incorporated, 
because developers are interested in improving the software, and the users in the improvements.
%
This demonstrates to project leaders (in communities, foundations,
governments, and companies) 
the importance to monitor LCOM4 and CBO metrics together with NM and LOC, 
thereby increasing their chances of forming a community of contributors 
around their software, further enhancing its quality.
Thus, projects should grow managing their complexity, keeping the new
members willingness to contribute. 

Our study differs from related work because we analyzed a large
sample of Free Software projects. Table \ref {table:statistics} shows how diverse our sample of
6,773 projects is. There are projects with thousands of modules (7,177),
millions of lines of code (2,983,103), large structural complexity (4,940),
several hundred members (288), and millions of downloads (9,000,000).
This sample was based on well-defined criteria and the number of projects
involved provided us with statistical confidence in the results.

Nevertheless, this study has some limitations that motivate future work.
%
Our sample is restricted to projects written in C available at
SourceForge.net and our analyses to a limited set of metrics. 
In the future, we will include projects from other repositories, and extend 
this study to other source code metrics and programming languages such as C++ 
and Java. Furthermore, widely known projects such as GNU/Linux and Firefox should be 
included in studies of this kind, for their metrics may signal references 
needing to be followed.

Finally, we acknowledge that source code metrics are not the only
variables capable of influencing attractiveness.
%
Our previous work has identified that things such as the restrictiveness of
the license, type of project, software life-cycle stage, and intended audience 
are all capable of influencing attractiveness \cite{Santos2010}.
%
At first sight, assuming that all these variables from our previous study are
independent from the source code metrics we studied here, roughly 40\% of attractiveness variance would be then explained.
%
However, including variables in an equation in a statistically sound manner is not 
straightforward. Accordingly, there is a need to further identify the interaction 
between that set of variables with the ones reported in this study.

%\vspace{-0.5em}
\section{Acknowledgments}
The authors of this paper are supported by CNPQ, FAPESP, and the Qualipso project. 
This research has been developed in the USP FLOSS Competence Center and the 
authors would like to thank Claudia Melo, Lucianna Almeida, Joenio Costa, 
Beraldo Leal, and Nelson Lago for their contributions.

%\scalefont{0.9}
\bibliographystyle{sbc}
\bibliography{../myReferences}

\end{document}
